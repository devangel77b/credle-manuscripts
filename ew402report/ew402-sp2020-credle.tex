\documentclass[onecolumn,10pt]{IEEEtran}

\usepackage{amsmath,amsfonts,amssymb}
\usepackage{graphicx}
%\usepackage{marginnote} % for editorial use
\usepackage{sidenotes} % for editorial use
\usepackage[dvipsnames,svgnames]{xcolor}
\usepackage{svg,svg-extract}
\usepackage{booktabs}

\usepackage{siunitx}
\DeclareSIUnit\foot{ft}
\DeclareSIUnit\pound{lb}
\DeclareSIUnit\ounce{oz}
\DeclareSIUnit\inch{in}
\DeclareSIUnit\rpm{rpm}
\DeclareSIUnit\fahrenheit{F}
\DeclareSIUnit\bit{bit}
\DeclareSIUnit\byte{B}

\newcommand{\myroot}{../}
\newcommand{\Later}{\textbf{Later.}}
\newcommand{\Calypteanna}{\emph{Calypte anna}}
\newcommand{\Canna}{\emph{C.~anna}}
\newcommand{\MATLAB}{Matlab}
\newcommand{\Matlab}{Matlab}

\usepackage{hyperref}
\hypersetup{
  colorlinks=true,
  linkcolor=violet,
  urlcolor=blue,
  citecolor=blue}

\usepackage[plain]{fancyref}
\renewcommand{\freffigname}{Fig.}
\renewcommand{\Freffigname}{Fig.} 
\renewcommand{\freftabname}{Table}
\renewcommand{\Freftabname}{Table}
\frefformat{plain}{\fancyrefeqlabelprefix}{(#1)} 
\Frefformat{plain}{\fancyrefeqlabelprefix}{(#1)} 

\usepackage{listings}
\lstset{
	basicstyle=\ttfamily,
	columns=fullflexible,
	showstringspaces=false
}
\lstdefinestyle{mbedC}{
	language=C,
	basicstyle=\ttfamily,
	keywordstyle=\color{blue}\ttfamily,
	stringstyle=\color{magenta}\ttfamily,
	commentstyle=\color{green}\ttfamily,
	directivestyle=\color{red}\ttfamily,
 	morecomment=[l][\color{red}]{\#},
	columns=fullflexible,
	showstringspaces=false,
%	frame=single
}
\lstdefinestyle{usnaMatlab}{
	basicstyle=\ttfamily,
	keywordstyle=\color{blue}\ttfamily,
	stringstyle=\color{magenta}\ttfamily,
	commentstyle=\color{green}\ttfamily,
 	morecomment=[l][\color{red}]{\#},
	columns=fullflexible,
	showstringspaces=false,
	language=Matlab
%	%backgroundcolor=\color{lightgray},
%	frame=single
}

\usepackage[noadjust]{cite} % For IEEE references





\title{Autonomous drone racing}
\author{MIDN 1/C A.~Credle and Asst. Prof. D.~Evangelista\thanks{Author is with the Department of Weapons, Robotics, and Control Engineering at the United States Naval Academy. Address for correspondence: \emph{m201260@usna.edu}}}
\date{\today}


% For EW495 title page
\usepackage{titlepage4956}
\coursenumber{EW402}
\student{MIDN 1/C A. Credle}
\advisor{Assistant Professor D. Evangelista}
\coverpicture{\includegraphics[width=4.48in]{figures/titlepage.png}}


\begin{document}
\maketitlepage
\maketitle

\begin{abstract}
Autonomous racing drones are not only beneficial to the drone racing sport, but also in autonomous vehicles and in military drones flying through windows, chimneys, etc. Research in this field is often based around either race gate recognition or flight controls, but this project will incorporate both. We will create a process for a drone to modify its given flight path based on the visual recognition of a simplified, idealized gate in order to fly through the gate. The processor will identify the gate, locate it in relation to the drone’s current path, and modify the path to fly through the gate. I demonstrated this process by using physical experiments with a Tello quadrotor operating in four phases: (1) computer flight control with motion primitives; (2) offboard gate detection in test images; (3) reception of onboard video feed by offboard control computer; (4) combination of video feed and gate detection; and (5) successful navigation of the gate. The successful implementation of these steps on a slow platform like a Tello is a good in creating an autonomous racing drone at USNA; future work should scale up to faster platforms and more realistic gate conditions.
\end{abstract}

\begin{IEEEkeywords}
capstone, robotics, controls
\end{IEEEkeywords}


\section{Introduction}
\subsection{Background and motivation}
\IEEEPARstart{T}{he concept of drone racing} is straightforward: a group of people fly unmanned aerial vehicles (UAV) through gates, the first one to the finish line wins. The UAVs, or drones, come in a variety of sizes, ranging from an inch in diameter to over a foot, and are flown through communication with a remote controller. Races can take place in a variety of conditions including time of day and location. Gates can be configured in an unlimited number of patterns and often come in circular or rectangular form, though they are not limited to these shapes. Limitations are often placed on the power and size of drones that are raced, as owners are often expected to bring their own equipment \cite{redbull2018drone}. While the concept of racing is not new, drone racing is one of the fastest growing sports in the world \cite{condliffe2016is}. Its fame is quickly growing, and sports broadcasting companies like ESPN are signing contracts to broadcast competitions \cite{marshall2017espn}. There are no limitations on age, gender, ethnicity, or physical prowess, making drone racing available to everyone.
\begin{figure}[hb]
\begin{center}
\includegraphics[width=4.14in]{figures/fig1.png}
\end{center}
\caption{Two drones racing, from \cite{someone}}
\label{fig:1}
\end{figure}

The concept of autonomous drone racing is even newer, though its potential goes beyond that of its sport. In concept, a human drone racer goes through a series of cues and maneuvers to race the drone. If these cues and maneuvers are broken down into systematic steps, it is possible to automate the process and have the drone fly itself through the course. In modern autonomous vehicles, the specific location of the vehicle is often known, through GPS, a tracking system, etc. While this method of localization works for vehicles with static environments, it does not take into account the location of its surrounds, or the vehicles location in relation to specific objects. 

By mixing both an assumed path and gates for the drone to fly through, the exact location of the drone is not required; its relative location to the gates is all that is needed to fly a successful race. This concept will be critical for future autonomous systems, such as self-driving cars, that are required to navigate based on an assumed path mixed with the vehicles relative to the environmental objects or barriers (\fref{fig:1}) \cite{iot2018how}. For cars, the environmental objects would include lane lines, other vehicles, curbs, objects in the road, etc \cite{rayej2014how}. By integrating objects of the environment into the core of the navigation process, autonomous vehicles will be able to keep both passengers, pedestrians, and wildlife safe. Autonomous vehicles of the future must be able to navigate based on both an assumed path and a relative location in order to efficiently and safely navigate dynamic environments.  
\begin{figure}[hb]
\begin{center}
\includegraphics[width=3.90in]{figures/fig2.png}
\end{center}
\caption{An autonomous car sensing its environment, from \cite{someone}}
\label{fig:2}
\end{figure}

As the first autonomous drone-racing project at USNA, this will open the door to future drone research for midshipmen. EW281 and EW282 provide opportunities for midshipmen to experiment with drones on a hardware and flight-testing level, but this project will allow for future drone development on the software and autonomy level. Additionally, this research is only the first step in creating a fully autonomous racing drone that rivals a human’s performance. Future steps include path optimization, recognition with visual uncertainties, racer collision avoidance, etc. This research will be one small step in the larger picture of fully autonomous drone racing.

\subsection{Problem statement}
Given a three-dimensional flight path, drone accelerometer readings, and the image of a drone-racing gate in a three dimensional environment, this research intends to begin the process of developing autonomous drone flight through a drone-racing course. Given a three-dimensional path that does not fly through the gate, we will find and test the feasibility of a guidance system that creates a similar path that does. This system will be tested by placing a small quadrotor in multiple scenarios in relation to a race gate(s), with varying pre-determined paths, and directing the guidance system to fly the drone through the race gate(s). Success of this system is based on the ability to autonomously correct the path and fly the drone through the center of the gate without contact. This will be measured by the difference in distance between the path at its intersection with the gate, and the center of the gate. The competing objective will be to maintain the same trajectory at the intersection with the gate as if the path had not been modified. The final output of this project will a guidance system that could be implemented on any racing drone with a camera and IMU and fly through race gates autonomously.
\begin{figure}[t]
\begin{center}
\includegraphics[width=4in]{figures/fig3.png}
\end{center}
\caption{Depiction of flight path correction (original path in green, new path in blue, trajectory at gate intersection as red arrow, gate intersection point as red dot)}
\label{fig:3}
\end{figure}

\subsection{Literature review}
This paper is the one of the first developments of a drone autonomous flight program, so the background research encompasses papers that focus on multiple aspects of autonomous flight. The main categories that these papers include navigation, flight controls, and visual recognition. This research focuses on the visual recognition of the gate and navigation, but these flight controls is necessary for the background research because it affect the performance of navigation and is necessary for basic demonstration.

In order to understand the movement patterns of the drone, \cite{svacha2017improving} provides a set of equations that could be used to accurately model the induced drag and thrust forces. These equations were derived from proofs using properties of physics, and then the coefficients were identified through flight experimentation. This provides suitable estimations for the forces acting on the drone, which allows for control of the drone through the gate. This is connected to \cite{condliffe2016is}, which created a state space model that could accurately predict the movement of a small drone with aggressive flight. By determining these equations through modeling and live testing, these state equations can be coupled with the flight force equations from \cite{svacha2017improving} to form an accurate estimate of the drone’s location. Coupling these with a flight controller will allow the drone to have a basic understanding of its location within the general space. This research intends to use visual servoing to adjust a drone’s flight paths, and in order to understand the drone’s relation to the predetermined flight path, \cite{svacha2017improving} and \cite{loianno2017estimation} provide an accurate location.
    
While \cite{svacha2017improving} and \cite{loianno2017estimation} provide what is believed to be an accurate location of the drone in-flight, its location in relation to the preprogrammed flight path is not always certain. \cite{florence2018nanomap} provides the groundwork for a concept known as ``nano mapping'', which refers to modeling a 3D data structure depicting obstacles around the drone. This technique is different from a traditional mapping approach because traditional mapping is based off a common world frame. \cite{florence2018nanomap} is based solely on the frame of the drone, allowing for navigation based off the drones currently location, rather than its location relative to a known point. This concept is also crucial to the development of this research because the coupled flight path planning approach requires both the common world frame and the drone world frame.

Different methods for seeing the world around the drone exist, including range finding technology (used for altitude measurements), single camera, and multi-camera approaches. \cite{loianno2017estimation} is based on a single camera approach, while \cite{iot2018how} found that a three camera setup was optimal for uncertain terrains. \cite{zhilenkov2018use} is based on autonomous navigation of drones in a wooded area where the location of the trees is unknown before flight. The drone does not have a preprogramed path, but rather has a program to recognize key features along a route and to estimate the likelihood of needing to turn. While the racecourse for this research is assumed to be known, a single camera approach will be utilized for hardware simplicity; estimating the likelihood that the computer is seeing a gate, however, may be utilized in this research. If a neural network could be constructed to follow a path through a forest, a similar neural network could be formed to fly through an aerial path.
    
One of the main challenge that arises with flying through the gate is recognizing multiple gates. In small drone racing courses, it is likely that the drone will be able to see multiple gates within its field of view. \cite{jung2018perception} worked through this problem and found reasonable neural network parameters to mitigate this issue. By assuming the next gate in the racecourse would be the largest gate in the drone camera view (using a single camera approach), \cite{jung2018perception} removed the neural network layer of all image analysis and left only the recognition of the largest circle gate. By removing this process in the image analysis, \cite{jung2018perception} found a significant reduction in computation time while only reducing the accuracy of gate recognition by less than 10\% (\SIrange{464}{34}{\milli\second} reaction time, \SIrange{82.4}{75.5}{\percent}).
    
The other main challenge that arises after recognizing the proper gate to fly through is implementing a visual servoing program to direct the drone through the gate. \cite{jung2018direct} created a process of gate recognition that located the center of the gate in relation to the location of the center of the cameras view and redirecting the drone towards the center of the gate. This research is the closest process to the research in this paper, as it specifies a visual recognition process that is simplified into the computational level of an inflight processor.
    
While many of these papers are useful in the development of an autonomous drone, it should be noted that there lacks a consistency among drone researchers in the assumptions and underlying assumptions. As described earlier, some researchers utilize a single camera operating system, while others use multi-camera systems, and even some include range-finding technology. Many of the subjects of these papers delved into different subsections of drone autonomy, so it is reasonable that their methods varied. For research such as \cite{svacha2017improving}, \cite{loianno2017estimation}, and \cite{florence2018nanomap}, their focus was more for drone flight in general, so this critique does not apply as much as it does for projects such as \cite{zhilenkov2018use}, \cite{jung2018perception}, and \cite{jung2018direct}, which had varying sensor and visual capabilities. As the latter three projects had more to do with direct visual recognition, it would have been more advantageous for the autonomous drone community had the projects been embarked on in a similar manner.
    
These projects center around the research in this proposal in two ways: by giving basic flight control and navigational equations to use in baseline location estimations and flight controllers, and by providing simplistic approaches to visual recognition to model my approach. While my approach uses both a whole world frame and a drone view frame to form a novel approach to gate recognition, many of the approaches depicted in \cite{zhilenkov2018use}, \cite{jung2018perception}, and \cite{jung2018direct} can be used to model a realistic approach to melding the two frames. This project will take these approaches and form a simple and novel method to visual gate recognition and path modification.
 %\section{Introduction}
\section{Materials and methods}
The methods for proof-of-concept included three main phases: (1) Tello drone flight by computer; (2) Tello video streaming to computer; and (3) computer gate recognition. These three sections were developed separately in the Fall of 2019. In the Spring of 2020, the three phases were integrated in two additional phases: (4) combining gate detection with the live video feed from the Tello, and (5) full system integration to turn towards and navigate through the gate. Each is operated by a separate Python program (version 3.6.9, \url{https://www.python.org/}) on the same Linux system (Ubuntu 18.04LTS, \url{https://ubuntu.com/}). All code for this project is provided in the Github repository at \url{https://github.com/devangel77b/credle-2020-code}. 




\subsection{Drone racing hardware}
The original budgeted proposal called for four Tello drones, two Vortex 180 drones, and two sets of LED race gates. Upon receiving three Tello drones, it was found that their preprogrammed flight controls had a minimum movement requirement per command of \SI{20}{\milli\meter}. To best utilize the Tello drones and calculate the future path correction accuracy, a larger gate became more advantageous. The USNA School of Drones has a single, yellow-colored, \SI{5x5}{\foot} standard MultiGP drone racing race gate, so the smaller LED race gates were to be substituted, however, due to the global COVID-19 pandemic, these were not available. Testing was instead accomplished with yellow paper targets. The Vortex 180 drones have not been purchased as they were apart of the stretch goal for this research, which was not attempted due to the global COVID-19 pandemic.





\subsection{Phase 1: Tello drone first flight test}
For Tello drone flight, the first phase was to connect to the Tello drones flight controls in any manner possible to understand its capabilities. The easiest method was a connection over a smartphone through an app developed by DJI through the Tello’s Wi-Fi. Upon completion of this flight, focus was shifted to flight under computer control. A Python program (excerpt in \fref{fig:pseudocode1}, full code at \url{https://github.com/devangel77b/credle-2020-code}) was written utilizing the base code provided by DJI (\url{https://github.com/dji-sdk/Tello-Python}) to open a socket between the drone and Linux terminal. This connection was also established over the Tello drones Wi-Fi. The Tello is preprogramed to receive text commands over the socket and perform a take-off maneuver, landing maneuver, or translational/ rotational movements. Each of the preprogrammed commands were tested and verified to operate according to the DJI operational instructions \cite{tello-manual}.
%Give pseudocode here and link to repository containing the code. 
\begin{figure}
\caption{Code snippet for Tello drone flight from host terminal.}
\label{fig:pseudocode1}
\lstinputlisting[language=python]{code/tello-flight.py}
\end{figure}






\subsection{Phase 2: Video streaming from Tello test}
For Tello video streaming to a computer, a Python program was developed using the base code provided by DJI (\url{https://github.com/dji-sdk/Tello-Python}) to connect the Tello’s on-board camera to the same Linux system (excerpt in \fref{fig:pseudocode2}, full code at \url{https://github.com/devangel77b/credle-2020-code}). This worked in a similar manner to the flight controls by connecting over the Tello’s Wi-Fi using a socket connection, but differed in the required port connection. The connection allowed the computer to receive real-time flight video data, which was converted using an h264 decoder. The video stream was then integrated into a GUI for ease of viewing, and tested to ensure minimal connection lag and accurate color imaging.
%Give pseudocode here and link to repository containing the code. 
\begin{figure}
\caption{Code snippet for video streaming from a Tello.}
\label{fig:pseudocode2}
\lstinputlisting[language=python]{code/video-stream.py}
\end{figure}





\subsection{Phase 3: Gate recognition}
For computer gate recognition, the drone's onboard camera was utilized so that integration of the previous step was not required. A Python program (excerpt in \fref{fig:pseudocode1}, full code at \url{https://github.com/devangel77b/credle-2020-code}) was developed to view the real-time camera data and segment it based on HSV color segmenting \cite{bradski2008learning}. Appropriate values were found to segment out all colors other than the yellow/gold of the actual race gate, and visual race gate created using a sticky-note on a black board. The segmented sections of the video feed were then connected and analyzed for size. Small clumps were removed in an attempt to keep only the race gate. The remaining objects in the image were then analyzed for their number of edges. Knowing that the race gate has eight edges, any shape without 8 edges was removed. This left only the race gate in the segmented imaging.
%Give pseudocode here and link to repository containing the code. 
\begin{figure}
\caption{Code snippet for automatic gate recognition.}
\label{fig:pseudocode3}
\lstinputlisting[language=python]{code/gate-recognition.py}
\end{figure}

Due to the global COVID-19 pandemic, the final phases of system integration did not have access to the full sized \SI{5x5}{\foot} standard race gate, so a yellow piece of paper of approximately the same shade was used, trimmed with the same aspect ratio but scaled down. 





\subsection{Phase 4: Combining video streaming and gate detection}
In phase 4, I added gate detection to the Tello live video stream.  Addition of the gate detection function into the video streaming code was accomplished by modifying the \lstinline{TelloUI.videoLoop()} method in \lstinline{tello_control_ui.py} to include a call to a video processing function as follows:
\begin{lstlisting}[language=python]
            # read the frame for GUI show
                self.frame = self.tello.read()
                if self.frame is None or self.frame.size == 0:
                    continue 

            	#currently a np array under self.frame, thrown through processing after being pulled but before being put into the gui
		
                self.frame=self.video_process(self.frame)
\end{lstlisting}

The actual video processing works similar to Phase 2 code. The live image from the Tello was converted to hue-saturation-value colorspace using \lstinline{cv2.cvtColor()}, thresholded using \lstinline{cv2.inRange()}, and filtered using a \num{5x5} erosion operation implemented with \lstinline{cv2.erode()}. This provided a filtered binary image that could be processed later for centroid and size:
\begin{lstlisting}[language=python]
   def video_process(self, image):
	    #basic thresholding code pulled from gate recognition
	    #sees blue? 
	    
	    hsv = cv2.cvtColor(image, cv2.COLOR_BGR2HSV)

	    lower_red = np.array([85, 100, 94])
	    upper_red = np.array([100, 168, 255])

	    mask = cv2.inRange(hsv, lower_red, upper_red)
	    kernel = np.ones((5,5), np.uint8)
	    image = cv2.erode(mask, kernel)

	    return image
\end{lstlisting}





\subsection{Phase 5: Full system integration}
Final system integration was accomplished by adding autonomous control to the code from Phase 4. I added two additional attributes to the \lstinline{TelloUI} object, \lstinline{TelloUI.degree} and \lstinline{TelloUI.distance}, representing an increment to turn or move forward, respectively. The pseudocode to detect and navigate a gate then becomes:
\begin{lstlisting}
find binary image
compute moments for the binary image
if the image is not zero: 
  if the gate is centered:
    move forward
  else (the gate is left)
    turn left 
  else 
    turn right
else continue
\end{lstlisting}

In Python, this takes the form of a new \lstinline{TelloUI.video_analyze()} method that is called after the call to \lstinline{video_process()}:
\begin{lstlisting}[language=python]
    def video_analyze(self, image):
	    #finding center of mask

	    #im2,contours,hierarchy = cv.findContours(image, 1, 2)
	    #cnt = contours[0]
	    M=cv2.moments(image)

# if one of these is 0, skip
#paper in front of tello, move paper, tello moved
#if centerd for a few seconds, move forward
	    
	    if M['m00'] !=0:
		cx=int(M['m10']/M['m00'])
	    	cy=int(M['m01']/M['m00'])
	    	if cx>240 and cx<480:
			#print("middle")
			self.countl=0
			self.countr=0
			self.countm=self.countm+1
			if self.countm==200:
				self.tello.move_forward(.2)
				print("moving")
				self.countm=0
	    	elif cx<=240:
			#print("left")
			self.countm=0
			self.countr=0
			self.countl=self.countl+1
			if self.countl==200:
				self.tello.rotate_ccw(self.degree)
				print("moving")
				self.countm=0
			
		else:
			#print("right")
			self.countm=0
			self.countl=0
			self.countr=self.countr+1
			if self.countr==200:
				self.tello.rotate_cw(self.degree)
				print("moving")
				self.countr=0
	    else:
		print("no image")
\end{lstlisting}

 %\section{Materials and methods}
\section{Results}

\subsection{Phase 1: Tello drone first flight test}
Drone flight was successfully controlled by the computer and could be maintain for approximately \SI{4}{\minute} on a single battery. All pre-programmed maneuvers were tested and performed as described. \Fref{fig:4} depicts the right flip maneuver in action.
\begin{figure}
\begin{center}
\includegraphics[width=4.06in]{figures/fig4.png}
\end{center}
\caption{Combination of video frames filming a right flip maneuver of the Tello drone with computer controls. How many frame per second. Add scale bar and callouts.}
\label{fig:4}
\end{figure}

\subsection{Phase 2: Tello video stream test}
The Tello drone built-in camera was successfully connected to the Linux terminal. \Fref{fig:5} depicts the active connection between the two systems. 
\begin{figure}
\begin{center}
\includegraphics[width=3.96in]{figures/fig5.png}
\end{center}
\caption{Tello drone streaming video to Linux system. Add scale bar and callouts.}
\label{fig:5}
\end{figure}

\subsection{Phase 3: Gate recognition}
The Python program successfully recognized a drone race gate through color segmentation and edge recognition. The program both segmented out incorrect gate colors, objects too small to be the gate, and objects with too many or too few edges to be the gate. \Fref{fig:6} depicts both the segmented image and the original image with identified edges overlaid.
\begin{figure}
\begin{center}
\includegraphics[width=3.97in]{figures/fig6.png}
\end{center}
\caption{Gate recognition program segmenting for gate color and projecting edges on gate in non-segmented image.}
\label{fig:6}
\end{figure}

\subsection{Phase 4: Combining video steeaming and gate detection}
Combination of the Tello video streaming and image processing to recognize the gate using color thresholding was successful. \Fref{fig:7} depicts the improvised setup using a yellow piece of paper, necessary due to the global COVID-19 pandemic, as well as the program successfully detecting it. 
\begin{figure}
\begin{center}
\includegraphics[width=0.3\columnwidth]{figures/IMG_2235.png}
\includegraphics[width=0.3\columnwidth]{figures/IMG_2236.png}
\end{center}
\caption{Gate recognition program segmenting for gate color running online with live video feed from the Tello.}
\label{fig:7}
\end{figure}

\subsection{Phase 5: Final system integration}
Fully autonomous gate detection and navigation was successful. \Fref{fig:8} shows successive screenshots of the approach to a gate, and turning to a new gate position/
\begin{figure}
\begin{center}
\includegraphics[angle=-90,width=0.15\columnwidth]{figures/vlcsnap-2020-05-07-11h08m57s743.png}
\includegraphics[angle=-90,width=0.15\columnwidth]{figures/vlcsnap-2020-05-07-11h09m07s372.png}
\includegraphics[angle=-90,width=0.15\columnwidth]{figures/vlcsnap-2020-05-07-11h09m15s197.png}
\includegraphics[angle=-90,width=0.15\columnwidth]{figures/vlcsnap-2020-05-07-11h09m20s279.png}
\includegraphics[angle=-90,width=0.15\columnwidth]{figures/vlcsnap-2020-05-07-11h09m27s534.png}
\includegraphics[angle=-90,width=0.15\columnwidth]{figures/vlcsnap-2020-05-07-11h09m33s534.png}
\end{center}
\caption{Successful online detection and navigation of an improvised gate. (a-d) Tello makes steady approach to a gate; (e, f) Tello turns when gate is turned.}
\label{fig:8}
\end{figure}

 %\section{Results}
\section{Discussion}

\subsection{Successful first steps towards autonomous drone racing}
This work was initiated with the hope of taking some first steps towards autonomous drone racing at the US Naval Academy. I was successful in making a Tello drone autonomously detect and navigate through a small yellow paper gate. I proceeded in steps from making the Tello fly from an offboard computer, detecting the gate in imagery, then combining it with the Tello's onboard video feed, before finally integrating it with the flight control system via high level motion primitives to check for gate centroid and turn or move forward. This allows for the drones relative location to the gate to be calculated and integrated with the flight controls to pass through the gate.

\subsection{Study limitations}
Obviously, the use of color thresholding is a simplification that is not possible in standard autonomous drone racing league events, where gates are multicolored and possess logos and patterns as well as more complex lighting.  My code does not consider the possibility of multiple gates in view, though it is able to handle the lack of a gate in view (as occurs at the instant of passing through it). 

The Tello drone used moves quite slowly and thus the speed of offboard image processing and control are not an issue. At actual drone racing speeds, latency could be a major risk. In addition, other projects (e.g. MIDN 1/C Marcello's work) have shown that as speeds increase, the use of simplistic position-based path planning top level control with lower level autopilots is likely to become performance limited; in human drone race pilots this necessitates use of ACRO mode vice (automatic) LEVEL. 

My path planning is only the most simple case of moving towards the centroid, as is used in our EW309 Guided Design Class nerf gun turrets. A more complex race algorithm should attempt to select a fast line through the turn, using knowledge of ownship position, the gate, as well as the course to the next gate after clearing the current one. 

\subsection{Additional thoughts}
In conclusion, this research began to create a process for a drone to modify its flight path based on the visual recognition of racing gates. This is the first steps in the process to create a fully autonomous racing drone. Future steps will include path optimization and processing optimization. The process for this research includes identify the gate, locate it in relations to the drone, and modify the flight path to fly through the gate. The processor will have a visual feed, the three dimensional acceleration data of the drone’s flight, and the pre-determined three-dimensional path. The process will output a new three-dimensional path for the drone to follow in order to fly through the gate. This project demonstrates this process using physical experiments to refine and optimize the program and measure results. Success is measured by the accuracy of the new three-dimensional path as it passes through the gate. Accuracy is based on the difference in distance of the path and the center of the gate. 

 %\section{Discussion}


\section*{Acknowledgements}
I thank MIDN 1/C Ethan Marcello and the (spell out acronym) (KEF) Robotics team competing in the Lockhood Martin Alpha Pilot Challenge for helpful comments which were incorporated into this proposal. I thank 2Lt Canlas and 2Lt Cunniff for their advice and suggestions, and Alan Drew (Sail Loft) for providing the original \SI{5x5}{\foot} gate. USNA School of Drones is supported by Broering and by Boeing.  USNA Biomechanics, which also provided some gear, is supported by Lockheed Martin. 

\bibliographystyle{IEEEtran}
\bibliography{IEEEabrv,\myroot/references/credle}

\begin{IEEEbiography}[{\includegraphics[width=1in,height=1.25in,clip,keepaspectratio]{\myroot/figures/M201260.jpg}}]{Austin Credle} is a midshipman at the United States Naval Academy majoring in Robotics and Control Engineering. Upon graduation, he will train as a Navy Pilot. 
\end{IEEEbiography}
%
%\begin{IEEEbiography}[{\includegraphics[width=1in,height=1.25in,clip,keepaspectratio]{\myroot/figures/evangelista_d_prof.jpg}}]{Dennis Evangelista} raises guide dog puppies. 
%\end{IEEEbiography}

%\clearpage
%\appendix
%\renewcommand{\figurename}{Supplementary Figure}
%\renewcommand{\thefigure}{S\arabic{figure}}
%\section{Updated budget information}
% EW502 is bullshit.
\begin{table}[h]
\caption{Updated budget information}
\label{tab:1}
\begin{center}
\includegraphics[width=\columnwidth]{figures/budget1.png}
\end{center}
\end{table}
\begin{table}[h]
\caption{Updated part information}
\label{tab:2}
\begin{center}
\includegraphics[width=\columnwidth]{figures/budget2.png}
\end{center}
\end{table}

\end{document}
