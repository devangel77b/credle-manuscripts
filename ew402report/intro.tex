\section{Introduction}
\subsection{Background and motivation}
\IEEEPARstart{T}{he concept of drone racing} is straightforward: a group of people fly unmanned aerial vehicles (UAV) through gates, the first one to the finish line wins. The UAVs, or drones, come in a variety of sizes, ranging from an inch in diameter to over a foot, and are flown through communication with a remote controller. Races can take place in a variety of conditions including time of day and location. Gates can be configured in an unlimited number of patterns and often come in circular or rectangular form, though they are not limited to these shapes. Limitations are often placed on the power and size of drones that are raced, as owners are often expected to bring their own equipment \cite{redbull2018drone}. While the concept of racing is not new, drone racing is one of the fastest growing sports in the world \cite{condliffe2016is}. Its fame is quickly growing, and sports broadcasting companies like ESPN are signing contracts to broadcast competitions \cite{marshall2017espn}. There are no limitations on age, gender, ethnicity, or physical prowess, making drone racing available to everyone.
\begin{figure}[hb]
\begin{center}
\includegraphics[width=4.14in]{figures/fig1.png}
\end{center}
\caption{Two drones racing, from \cite{someone}}
\label{fig:1}
\end{figure}

The concept of autonomous drone racing is even newer, though its potential goes beyond that of its sport. In concept, a human drone racer goes through a series of cues and maneuvers to race the drone. If these cues and maneuvers are broken down into systematic steps, it is possible to automate the process and have the drone fly itself through the course. In modern autonomous vehicles, the specific location of the vehicle is often known, through GPS, a tracking system, etc. While this method of localization works for vehicles with static environments, it does not take into account the location of its surrounds, or the vehicles location in relation to specific objects. 

By mixing both an assumed path and gates for the drone to fly through, the exact location of the drone is not required; its relative location to the gates is all that is needed to fly a successful race. This concept will be critical for future autonomous systems, such as self-driving cars, that are required to navigate based on an assumed path mixed with the vehicles relative to the environmental objects or barriers (\fref{fig:1}) \cite{iot2018how}. For cars, the environmental objects would include lane lines, other vehicles, curbs, objects in the road, etc \cite{rayej2014how}. By integrating objects of the environment into the core of the navigation process, autonomous vehicles will be able to keep both passengers, pedestrians, and wildlife safe. Autonomous vehicles of the future must be able to navigate based on both an assumed path and a relative location in order to efficiently and safely navigate dynamic environments.  
\begin{figure}[hb]
\begin{center}
\includegraphics[width=3.90in]{figures/fig2.png}
\end{center}
\caption{An autonomous car sensing its environment, from \cite{someone}}
\label{fig:2}
\end{figure}

As the first autonomous drone-racing project at USNA, this will open the door to future drone research for midshipmen. EW281 and EW282 provide opportunities for midshipmen to experiment with drones on a hardware and flight-testing level, but this project will allow for future drone development on the software and autonomy level. Additionally, this research is only the first step in creating a fully autonomous racing drone that rivals a human’s performance. Future steps include path optimization, recognition with visual uncertainties, racer collision avoidance, etc. This research will be one small step in the larger picture of fully autonomous drone racing.

\subsection{Problem statement}
Given a three-dimensional flight path, drone accelerometer readings, and the image of a drone-racing gate in a three dimensional environment, this research intends to begin the process of developing autonomous drone flight through a drone-racing course. Given a three-dimensional path that does not fly through the gate, we will find and test the feasibility of a guidance system that creates a similar path that does. This system will be tested by placing a small quadrotor in multiple scenarios in relation to a race gate(s), with varying pre-determined paths, and directing the guidance system to fly the drone through the race gate(s). Success of this system is based on the ability to autonomously correct the path and fly the drone through the center of the gate without contact. This will be measured by the difference in distance between the path at its intersection with the gate, and the center of the gate. The competing objective will be to maintain the same trajectory at the intersection with the gate as if the path had not been modified. The final output of this project will a guidance system that could be implemented on any racing drone with a camera and IMU and fly through race gates autonomously.
\begin{figure}[t]
\begin{center}
\includegraphics[width=4in]{figures/fig3.png}
\end{center}
\caption{Depiction of flight path correction (original path in green, new path in blue, trajectory at gate intersection as red arrow, gate intersection point as red dot)}
\label{fig:3}
\end{figure}

\subsection{Literature review}
This paper is the one of the first developments of a drone autonomous flight program, so the background research encompasses papers that focus on multiple aspects of autonomous flight. The main categories that these papers include navigation, flight controls, and visual recognition. This research focuses on the visual recognition of the gate and navigation, but these flight controls is necessary for the background research because it affect the performance of navigation and is necessary for basic demonstration.

In order to understand the movement patterns of the drone, \cite{svacha2017improving} provides a set of equations that could be used to accurately model the induced drag and thrust forces. These equations were derived from proofs using properties of physics, and then the coefficients were identified through flight experimentation. This provides suitable estimations for the forces acting on the drone, which allows for control of the drone through the gate. This is connected to \cite{condliffe2016is}, which created a state space model that could accurately predict the movement of a small drone with aggressive flight. By determining these equations through modeling and live testing, these state equations can be coupled with the flight force equations from \cite{svacha2017improving} to form an accurate estimate of the drone’s location. Coupling these with a flight controller will allow the drone to have a basic understanding of its location within the general space. This research intends to use visual servoing to adjust a drone’s flight paths, and in order to understand the drone’s relation to the predetermined flight path, \cite{svacha2017improving, loianno2017estimation} provide an accurate location.
    
While \cite{svacha2017improving, loianno2017estimation} provide what is believed to be an accurate location of the drone in-flight, its location in relation to the preprogrammed flight path is not always certain. \cite{florence2018nanomap} provides the groundwork for a concept known as ``nano mapping'', which refers to modeling a 3D data structure depicting obstacles around the drone. This technique is different from a traditional mapping approach because traditional mapping is based off a common world frame. \cite{florence2018nanomap} is based solely on the frame of the drone, allowing for navigation based off the drones currently location, rather than its location relative to a known point. This concept is also crucial to the development of this research because the coupled flight path planning approach requires both the common world frame and the drone world frame.

Different methods for seeing the world around the drone exist, including range finding technology (used for altitude measurements), single camera, and multi-camera approaches. \cite{loianno2017estimation} is based on a single camera approach, while \cite{iot2018how} found that a three camera setup was optimal for uncertain terrains. \cite{zhilenkov2018use} is based on autonomous navigation of drones in a wooded area where the location of the trees is unknown before flight. The drone does not have a preprogramed path, but rather has a program to recognize key features along a route and to estimate the likelihood of needing to turn. While the racecourse for this research is assumed to be known, a single camera approach will be utilized for hardware simplicity; estimating the likelihood that the computer is seeing a gate, however, may be utilized in this research. If a neural network could be constructed to follow a path through a forest, a similar neural network could be formed to fly through an aerial path.
    
One of the main challenge that arises with flying through the gate is recognizing multiple gates. In small drone racing courses, it is likely that the drone will be able to see multiple gates within its field of view. \cite{jung2018perception} worked through this problem and found reasonable neural network parameters to mitigate this issue. By assuming the next gate in the racecourse would be the largest gate in the drone camera view (using a single camera approach), \cite{jung2018perception} removed the neural network layer of all image analysis and left only the recognition of the largest circle gate. By removing this process in the image analysis, \cite{jung2018perception} found a significant reduction in computation time while only reducing the accuracy of gate recognition by less than 10\% (\SIrange{464}{34}{\milli\second} reaction time, \SIrange{82.4}{75.5}{\percent}).
    
The other main challenge that arises after recognizing the proper gate to fly through is implementing a visual servoing program to direct the drone through the gate. \cite{jung2018direct} created a process of gate recognition that located the center of the gate in relation to the location of the center of the cameras view and redirecting the drone towards the center of the gate. This research is the closest process to the research in this paper, as it specifies a visual recognition process that is simplified into the computational level of an inflight processor.
    
While many of these papers are useful in the development of an autonomous drone, it should be noted that there lacks a consistency among drone researchers in the assumptions and underlying assumptions. As described earlier, some researchers utilize a single camera operating system, while others use multi-camera systems, and even some include range-finding technology. Many of the subjects of these papers delved into different subsections of drone autonomy, so it is reasonable that their methods varied. For research such as \cite{svacha2017improving, loianno2017estimation, florence2018nanomap}, their focus was more for drone flight in general, so this critique does not apply as much as it does for projects such as \cite{zhilenkov2018use, jung2018perception, jung2018direct}, which had varying sensor and visual capabilities. As the latter three projects had more to do with direct visual recognition, it would have been more advantageous for the autonomous drone community had the projects been embarked on in a similar manner.
    
These projects center around the research in this proposal in two ways: by giving basic flight control and navigational equations to use in baseline location estimations and flight controllers, and by providing simplistic approaches to visual recognition to model my approach. While my approach uses both a whole world frame and a drone view frame to form a novel approach to gate recognition, many of the approaches depicted in \cite{zhilenkov2018use, jung2018perception, jung2018direct} can be used to model a realistic approach to melding the two frames. This project will take these approaches and form a simple and novel method to visual gate recognition and path modification.
